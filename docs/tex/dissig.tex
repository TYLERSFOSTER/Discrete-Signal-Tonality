\documentclass[11pt, a4paper]{article}

 \usepackage[left=2.5cm, right=2.5cm, top=2cm, bottom=2.5cm]{geometry}
\usepackage{amsmath}
\usepackage{amssymb}
\usepackage{amsfonts}

\usepackage{hyperref}
\usepackage{tikz}

\newcommand{\ExternalLink}{%
    \tikz[x=1.2ex, y=1.2ex, baseline=-0.05ex]{% 
        \begin{scope}[x=1ex, y=1ex]
            \clip (-0.1,-0.1) 
                --++ (-0, 1.2) 
                --++ (0.6, 0) 
                --++ (0, -0.6) 
                --++ (0.6, 0) 
                --++ (0, -1);
            \path[draw, 
                line width = 0.5, 
                rounded corners=0.5] 
                (0,0) rectangle (1,1);
        \end{scope}
        \path[draw, line width = 0.5] (0.5, 0.5) 
            -- (1, 1);
        \path[draw, line width = 0.5] (0.6, 1) 
            -- (1, 1) -- (1, 0.6);
        }
    }
 
 \newcommand{\Chi}{\mathrm{X}}

% Define the new "Problem" environment
\newtheorem{problem}{Problem}[subsection]

\begin{document}

\begin{section}{[...]}

\begin{subsection}{[...]}
In any setting where we have a monoid $M$ acting on a set $S$, and a subset $\Chi\subset M$, we can form the graph $\Gamma^{S/\Chi}_{\bullet}$ with
$$
\Gamma^{S/\Chi}_{0}:=\ S\text{\ \ \ \ \ \ and\ \ \ \ \ \ }\Gamma^{S/\Chi}_{1}:=\ \Chi\times S,
$$
where the *source* and *target* maps $$\partial^{1}_0,\partial^{1}_1:\ \Gamma^{S/\Chi}_1\!\!\longrightarrow\Gamma^{S/\Chi}_{0}$$ that interpret each pair $(\chi,\ s)\in \Chi\times S=\Gamma^{S/\Chi}_1$ as the arrow $$s\ \!\xrightarrow{\ \ \ \chi\ \ \ }\ \!\chi s.$$

In general, it's hard to "*zoom out from* $\Gamma^{S/\Chi}_{\bullet}$" to say anything substantuve about the large-scale structure of this graph. That said, there are many special casses where we can say a lot about the strucutre of this graph. In the special case that 
- $S=\mathbb{Z}/\ell\mathbb{Z}$, the set of integers modulo some positive integer $\ell$,
- $M=\mathbb{Z}/\ell\mathbb{Z}$ equipped with its mutliplicative structure,

we can say quite a bit about the strucutre of $\Gamma^{S/\Chi}_{\bullet}$. The positive integer $\ell$ has unique prime factorization $$\ell\ =\ p_{1}^{e_1}p_{2}^{e_2}\cdots p_{r}^{e_r}$$ The graph $\Gamma^{S/\Chi}_{\bullet}$ contains a subgraph $\text{Div}(\ell)_{\bullet}$ with $$\text{Div}(\ell)_{0}:=\{d\in\mathbb{Z}_{>0}:d\text{\ divides\ }\ell\}$$ and $$\text{Div}(\ell)_{1}:=\ \{d\xrightarrow{\ \ \ p\ \ }pd:\ pd\text{\ divides\ }\ell,\ p\text{\ prime}\}$$
\end{subsection}
\end{section}


\bibliographystyle{alpha}
\bibliography{../bib/dissig}

\end{document}


